%!TEX TS-program = xelatex
%!TEX encoding = UTF-8 Unicode
% Awesome CV LaTeX Template for CV/Resume
%
% This template has been downloaded from:
% https://github.com/posquit0/Awesome-CV
%
% Author:
% Claud D. Park <posquit0.bj@gmail.com>
% http://www.posquit0.com
%
% Template license:
% CC BY-SA 4.0 (https://creativecommons.org/licenses/by-sa/4.0/)
%


%-------------------------------------------------------------------------------
% CONFIGURATIONS
%-------------------------------------------------------------------------------
% A4 paper size by default, use 'letterpaper' for US letter
\documentclass[11pt, a4paper]{awesome-cv}
\usepackage[T1]{fontenc}
\usepackage[utf8]{inputenc}

% Configure page margins with geometry
\geometry{left=1.4cm, top=.8cm, right=1.4cm, bottom=1.8cm, footskip=.5cm}

% Color for highlights
% Awesome Colors: awesome-emerald, awesome-skyblue, awesome-red, awesome-pink, awesome-orange
%                 awesome-nephritis, awesome-concrete, awesome-darknight
\colorlet{awesome}{awesome-red}
% Uncomment if you would like to specify your own color
% \definecolor{awesome}{HTML}{CA63A8}

% Colors for text
% Uncomment if you would like to specify your own color
% \definecolor{darktext}{HTML}{414141}
% \definecolor{text}{HTML}{333333}
% \definecolor{graytext}{HTML}{5D5D5D}
% \definecolor{lighttext}{HTML}{999999}
% \definecolor{sectiondivider}{HTML}{5D5D5D}

% Set false if you don't want to highlight section with awesome color
\setbool{acvSectionColorHighlight}{true}

% If you would like to change the social information separator from a pipe (|) to something else
\renewcommand{\acvHeaderSocialSep}{\quad\textbar\quad}
\newcommand{\githublogo}{\faGithub}
\newcommand{\gitlogo}{\includegraphics[height=0.25cm]{logos/git.png}}
\newcommand{\cpplogo}{\includegraphics[height=0.25cm]{logos/C++.png}}
\newcommand{\clogo}{\includegraphics[height=0.25cm]{logos/C.png}}
\newcommand{\pythonlogo}{\includegraphics[height=0.25cm]{logos/Python.png}}
\newcommand{\numpylogo}{\includegraphics[height=0.25cm]{logos/numpy.png}}
\newcommand{\scipylogo}{\includegraphics[height=0.25cm]{logos/scipy.png}}
\newcommand{\matplotliblogo}{\includegraphics[height=0.25cm]{logos/matplotlib.png}}
\newcommand{\keraslogo}{\includegraphics[height=0.25cm]{logos/keras.png}}
\newcommand{\mathematicalogo}{\includegraphics[height=0.25cm]{logos/Mathematica.png}}
\newcommand{\tensorflowlogo}{\includegraphics[height=0.25cm]{logos/tensoflow.png}}
\newcommand{\linuxlogo}{\includegraphics[height=0.25cm]{logos/linux.png}}
\newcommand{\gnulogo}{\includegraphics[height=0.25cm]{logos/gnu.png}}
\newcommand{\latexlogo}{\includegraphics[height=0.25cm]{logos/LaTeX.png}}
\newcommand{\bashlogo}{\includegraphics[height=0.25cm]{logos/bash.png}}
\newcommand{\fortranlogo}{\includegraphics[height=0.25cm]{logos/Fortran.png}}
\newcommand{\vscodelogo}{\includegraphics[height=0.25cm]{logos/vscode.png}}
\newcommand{\vimlogo}{\includegraphics[height=0.25cm]{logos/Vim.png}}
\newcommand{\emacslogo}{\includegraphics[height=0.25cm]{logos/emacs.png}}
\newcommand{\cmakelogo}{\includegraphics[height=0.25cm]{logos/Cmake.png}}
\newcommand{\sqlitelogo}{\includegraphics[height=0.25cm]{logos/Sqlite.png}}
\newcommand{\numbalogo}{\includegraphics[height=0.25cm]{logos/Numba.png}}
\newcommand{\slurmlogo}{\includegraphics[height=0.25cm]{logos/slurm.png}}
\newcommand{\pbslogo}{\includegraphics[height=0.25cm]{logos/pbs.png}}
% \newcommand{\githublogo}{\textcolor{teal}{\textbf{github}}}
% \newcommand{\inspire}{\aiicon{inspire}}
\newcommand{\inspire}{\textcolor{awesome-red}{\textbf{Inspire}}}
\newcommand{\adalogo}{\includegraphics[height=0.25cm]{logos/ada_logo2.png}}
\newcommand{\doorslogo}{\includegraphics[height=0.25cm]{logos/doors_logo.png}}
\newcommand{\rtclogo}{\includegraphics[height=0.25cm]{logos/rtc_logo.png}}
\newcommand{\windowslogo}{\includegraphics[height=0.25cm]{logos/windows_logo.png}}
\newcommand{\officelogo}{\includegraphics[height=0.25cm]{logos/microsoftoffice_logo.png}}
\newcommand{\qtlogo}{\includegraphics[height=0.25cm]{logos/qt.png}}
\newcommand{\gnatlogo}{\includegraphics[height=0.25cm]{logos/gnatstudio_logo.png}}
\newcommand{\wiresharklogo}{\includegraphics[height=0.25cm]{logos/wireshark_logo.png}}
\newcommand{\xmllogo}{\includegraphics[height=0.25cm]{logos/xml_logo.png}}

\usepackage{academicons}
\usepackage{fontawesome5}
\usepackage{tcolorbox}
\usepackage{xcolor}

%-------------------------------------------------------------------------------
%	PERSONAL INFORMATION
%	Comment any of the lines below if they are not required
%-------------------------------------------------------------------------------
% Available options: circle|rectangle,edge/noedge,left/right
% \photo{./examples/profile.png}
\name{Niccol\`o}{Laurenti}
% \name{Prova}{Prova}
\position{Ph.D.\ Graduate in Particle Physics~~~·~~~Software Developer}
% \position{Site Reliability Engineer{\enskip\cdotp\enskip}Software Architect}
% \address{via Leonida Rech 80, Rome, 00156, Italy}

\mobile{(+39) 3382971956}
\email{niclaurenti@gmail.com}
% \dateofbirth{March 17th, 1997}
\linkedin{niccolo-laurenti}
\github{niclaurenti}
\homepage{https://niclaurenti.github.io}
\orcid{0009-0001-0718-0409}

% \gitlab{gitlab-id}
% \stackoverflow{SO-id}{SO-name}
% \twitter{@twit}
% \skype{skype-id}
% \reddit{reddit-id}
% \medium{medium-id}
% \kaggle{kaggle-id}
% \hackerrank{hackerrank-id}
% \googlescholar{googlescholar-id}{name-to-display}
%% \firstname and \lastname will be used
% \googlescholar{googlescholar-id}{}
% \extrainfo{extra information}

% \quote{``Be the change that you want to see in the world."}


%-------------------------------------------------------------------------------
\begin{document}

% Print the header with above personal information
% Give optional argument to change alignment(C: center, L: left, R: right)
\makecvheader

% Print the footer with 3 arguments(<left>, <center>, <right>)
% Leave any of these blank if they are not needed
\makecvfooter
  {\today}
  {Niccol\`o Laurenti~~~·~~~Curriculum Vitae}
  {\thepage}


%-------------------------------------------------------------------------------
%	CV/RESUME CONTENT
%	Each section is imported separately, open each file in turn to modify content
%-------------------------------------------------------------------------------

% %-------------------------------------------------------------------------------
%	SECTION TITLE
%-------------------------------------------------------------------------------
\cvsection{Summary}


%-------------------------------------------------------------------------------
%	CONTENT
%-------------------------------------------------------------------------------
\begin{cvparagraph}

%---------------------------------------------------------
Ph.D.\ researcher at the University of Milan specialised in applying artificial intelligence to particle physics.
I have experience working with different programming languages, in particular with C++ and Python.
I have hands-on experience with various machine learning tools like Keras and Tensorflow.
Passionate about the field of computer science and open to opportunities in industry to further improve my skills.
\end{cvparagraph}
%-------------------------------------------------------------------------------
%	SECTION TITLE
%-------------------------------------------------------------------------------
\cvsection{Personal Informations}


%-------------------------------------------------------------------------------
%	CONTENT
%-------------------------------------------------------------------------------
\begin{cvskills}

    %---------------------------------------------------------
    \cvskill
        {Birth} % Category
        {1997, Rome, Italy} % Skills

    %---------------------------------------------------------
    % \cvskill
    %     {Place of birth} % Category
    %     {Rome, Italy} % Skills

    %---------------------------------------------------------
    \cvskill
        {Citizenship} % Category
        {Italian} % Skills
    
    %---------------------------------------------------------
    \cvskill
        {Languages} % Category
        {Italian (native language), English (fluent)} % Skills

    %---------------------------------------------------------
\end{cvskills}
\cvsection{Experience}


%-------------------------------------------------------------------------------
%	CONTENT
%-------------------------------------------------------------------------------
\begin{cventries}

%---------------------------------------------------------
  \cventry
  {Researcher in Theoretical Particle Physics at the University of Milan and INFN}
  {Ph.D.\ Researcher}
  {Milan, Italy}
  {Oct.\ 2021 - Sept.\ 2024}
  {
      \begin{cvitems} % Description(s) of tasks/responsibilities
          \item Worked under the supervision of \href{https://inspirehep.net/authors/1009661?ui-citation-summary=true}{Prof.\ Stefano Forte} in the \href{https://nnpdf.mi.infn.it}{\textbf{NNPDF}} collaboration 
          as a developer of the \texttt{NNPDF} code \href{https://github.com/NNPDF}{\githublogo}.
          \item Developed techniques and computational programs that utilize artificial intelligence for 
          investigating the internal structure of the proton analysing experimental data collected at \href{https://home.cern}{\textbf{CERN}}.
          \item Developed programs for solving the so-called \href{https://en.wikipedia.org/wiki/DGLAP_evolution_equations}{DGLAP equations}, a linear system of integro-differential equations, with numerical techniques.
          \item Published research results in various papers and presented them in conferences.
          \item[] \textbf{\textcolor{awesome-red}{Tec}hnologies}: \pythonlogo{}~Python, \numpylogo{}~Numpy, \scipylogo{}~Scipy, \matplotliblogo{}~Matplotlib{}, \keraslogo{}~Keras, \tensorflowlogo{}~Tensorflow, \fortranlogo{}~Fortran, \bashlogo{}~Bash, \gitlogo{}~Git, \githublogo{}~Github, \mathematicalogo{}~Mathematica, \linuxlogo{}~Linux, \faApple{}~MacOS, \vscodelogo{}~VS Code, \vimlogo{}~Vim, \latexlogo{}~Latex, \sqlitelogo{}~SQLite
      \end{cvitems}
      }

    \cventry
{Researcher in Theoretical Particle Physics at the University of Rome ``La Sapienza''}
{Undergraduate Researcher}
{Rome, Italy}
{Mar.\ 2021 - Oct.\ 2021}
{
      \begin{cvitems} % Description(s) of tasks/responsibilities
        \item Worked under the supervision of \href{https://inspirehep.net/authors/1058479?ui-citation-summary=true}{Dr.\ Marco Bonvini} to develop theoretical methods and computational programs for producing high-precision theoretical predictions in particle physics.
        \item Focused on describing experimental data of electron-proton collisions, collected at the particle accelerators \href{https://en.wikipedia.org/wiki/HERA_(particle_accelerator)}{\textbf{HERA}} and \href{https://en.wikipedia.org/wiki/SLAC_National_Accelerator_Laboratory}{\textbf{SLAC}}.
        \item Wrote from zero the \texttt{C++} library \texttt{Adani} \href{https://github.com/niclaurenti/adani}{\githublogo}, resulting in a published paper and presentations at conferences.
        \item[] \textbf{\textcolor{awesome-red}{Tec}hnologies}: \cpplogo{}~C++, \gnulogo{}~GSL, \mathematicalogo{}~Mathematica, \linuxlogo{}~Linux, \bashlogo{}~Bash, \cmakelogo{}~CMake, \emacslogo{}~Emacs, \latexlogo{}~Latex
      \end{cvitems}
    }

\end{cventries}
%-------------------------------------------------------------------------------
%	SECTION TITLE
%-------------------------------------------------------------------------------
\cvsection{Skills}


%-------------------------------------------------------------------------------
%	CONTENT
%-------------------------------------------------------------------------------
\begin{cvskills}

    %---------------------------------------------------------
    \cvskill
    {Programming} % Category
    {C, C++, Python, Java, Ada, Fortran, Bash, XML, YAML, JSON, CMake} % Skills

    %---------------------------------------------------------
    \cvskill
    {Operating systems} % Category
    {Linux, MacOS, Windows} % Skills

    %---------------------------------------------------------
    \cvskill
    {Code editors} % Category
    {VS Code, Qt Creator, Apache NetBeans, Gnat Studio, Emacs, Vim, Nano} % Skills

    %---------------------------------------------------------
    \cvskill
    {Version control sysytems} % Category
    {Git, Github, Gitlab, Bitbucket, IBM RTC} % Skills

    %---------------------------------------------------------
    % \cvskill
    % {C++ libraries} % Category
    % {STL, GSL, Pybind11, Boost}
    
    %---------------------------------------------------------
    \cvskill
    {Python packages} % Category
    {Numpy, Scipy, Matplotlib, Multiprocessing, Numba, Pandas, Keras, Tensorflow, SQLite}

    %---------------------------------------------------------
    \cvskill
    {Jobs schedulers} % Category
    {Slurm, PBS}
    
    %---------------------------------------------------------
    \cvskill
    {Scientific programs} % Category
    {Matlab, Wolfram Mathematica} % Skills

    %---------------------------------------------------------
    \cvskill
    {Writing} % Category
    {Latex, Markdown, Microsoft Office} % Skills

    %---------------------------------------------------------
\end{cvskills}
%-------------------------------------------------------------------------------
%	SECTION TITLE
%-------------------------------------------------------------------------------
\cvsection{Education}


%-------------------------------------------------------------------------------
%	CONTENT
%-------------------------------------------------------------------------------
\begin{cventries}

    %---------------------------------------------------------
    \cventry
        {University of Milan} % Institution
        {Ph.D.\ in Physics} % Degree
        {Milan, Italy} % Location
        {Oct.\ 2021 - current} % Date(s)
        {
        \begin{cvitems} % Description(s) bullet points
            \item Field of study: Theoretical Particle Physics, Computational Physics.
            \item Graduating in fall 2024.
        \end{cvitems}
        }

    %---------------------------------------------------------
    \cventry
        {University of Rome ``La Sapienza''} % Institution
        {M.S.\ in Physics} % Degree
        {Rome, Italy} % Location
        {Sep.\ 2019 - Oct.\ 2021} % Date(s)
        {
        \begin{cvitems} % Description(s) bullet points
            \item Field of study: Theoretical Particle Physics.
            \item Grade: 110/110 (cum laude).
            \item Thesis: \href{https://arxiv.org/pdf/2401.12139.pdf}{\emph{Construction of a next-to-next-to-next-to-leading order approximation for heavy flavour production in deep inelastic scattering with quark masses}}.
        \end{cvitems}
        }

    %---------------------------------------------------------
    \cventry
        {University of Rome ``La Sapienza''} % Institution
        {B.S.\ in Physics} % Degree
        {Rome, Italy} % Location
        {Sep.\ 2016 - Nov.\ 2019} % Date(s)
        {
        \begin{cvitems} % Description(s) bullet points
            \item Grade: 110/110 (cum laude).
            \item Thesis: \emph{Particle identification with the time of flight method and applications to the CMS experiment}.
        \end{cvitems}
        }

    %---------------------------------------------------------
\end{cventries}
\cvsection{Publications}

\begin{cvhonors}

    %---------------------------------------------------------
    % \cvhonor
    % {\href{}{Implementation of DIS at N$^3$LO for PDF determination}} % Category
    % {A.\ Barontini, M.\ Bonvini, N.\ Laurenti, \emph{Eur.\ Phys.\ J.\ C}} % Skills
    % {}
    % {2024}

    \cvhonor
    {LO, NLO, and NNLO Parton Distributions for LHC Event Generators} % Category
    {J.~Cruz-Martinez, S.~Forte, N.~Laurenti, T.~R.~Rabemananjara, J.~Rojo, \emph{Eur.\ Phys.\ J.\ C}} % Skills
    {\vspace{0.35cm}\href{https://inspirehep.net/literature/2800507}{\aiicon{inspire}}}
    {\vspace{0.35cm}2024}

    \cvhonor
    {NNPDF4.0 aN$^3$LO PDFs with QED corrections} % Category
    {A.~Barontini, N.~Laurenti, J.~Rojo, \emph{Contribution to \href{https://inspirehep.net/conferences/2667502?ui-citation-summary=true}{DIS2024}}} % Skills
    {\href{https://inspirehep.net/literature/2794583}{\aiicon{inspire}}}
    {2024}
    
    \cvhonor
    {The Path to N$^3$LO Parton Distributions} % Category
    {The NNPDF Collaboration, R.~D.~Ball et al., \emph{Eur.\ Phys.\ J.\ C}} % Skills
    {\href{https://inspirehep.net/literature/2762925}{\aiicon{inspire}}}
    {2024}
    
    \cvhonor
    {Determinantion of the theory uncertainties from missing higher orders on NNLO parton distributions with percent accuracy} % Category
    {The NNPDF Collaboration, R.~D.~Ball et al., \emph{Eur.\ Phys.\ J.\ C}} % Skills
    {\vspace{0.35cm}\href{https://inspirehep.net/literature/2749502}{\aiicon{inspire}}}
    {\vspace{0.35cm}2024}
    
    \cvhonor
    {Photons in the proton: implications for the LHC} % Category
    {The NNPDF Collaboration, R.~D.~Ball et al., \emph{Eur.\ Phys.\ J.\ C}} % Skills
    {\href{https://inspirehep.net/literature/2747770}{\aiicon{inspire}}}
    {2024}

    \cvhonor
    {Inclusion of QED corrections in PDFs fits}
    {N.~Laurenti, \textit{Nucl.\ Part.\ Phys.\ Proc.}}
    {\href{https://doi.org/10.1016/j.nuclphysbps.2023.11.013}{\aiicon{doi}}}
    {2023}
    
    \cvhonor
    {Approximating missing higher-orders in transverse momentum distributions using resummations}
    {N.~Laurenti, T.~R.~Rabemananjara, and R.~Stegeman, \emph{Contribution to \href{https://inspirehep.net/conferences/1914506?ui-citation-summary=true}{DIS2022}}}
    {\vspace{0.35cm}\href{https://inspirehep.net/literature/2122473}{\aiicon{inspire}}}
    {\vspace{0.35cm}2022}

    %---------------------------------------------------------
\end{cvhonors}
\cvsection{Talks}
\begin{cvhonors}

    %---------------------------------------------------------
    \cvhonor
    {Evidence of intrinsic charm quarks in the proton} % Category
    {Mainz, Germany}
    {\href{https://indico.him.uni-mainz.de/event/171/contributions/1438/}{MENU23}} % Skills
    % {20 Oct. 2023}
    {2023}

    \cvhonor
    {Including QED corrections in PDF fits: The NNPDF4.0QED PDF set} % Category
    {Durham, UK}
    {\href{https://conference.ippp.dur.ac.uk/event/1128/contributions/6473/}{QCD@LHC23}} % Skills
    % {5 Sept 2023}
    {2023}

    \cvhonor
    {Inclusion of QED corrections in PDFs: The NNPDF4.0QED PDF set} % Category
    {Montpellier, France}
    {\href{https://qcd23.sciencesconf.org/}{QCD23}} % Skills
    % {10 July 2023}
    {2023}

    \cvhonor
    {Construction of a third order approximation for heavy flavour production in deep inelastic scattering} % Category
    {Milan, Italy}
    {\vspace{0.35cm}\href{https://indico.cern.ch/event/1095418/contributions/4656984/}{MCM 2021}} % Skills
    % {21 Dec 2021}
    {\vspace{0.35cm}2021}
    
    


    %---------------------------------------------------------
\end{cvhonors}
\cvsection{Teaching activity}

\begin{cvhonors}

    %---------------------------------------------------------
    \cvhonor
    {Co-supervisor of a Bachelor thesis} % Category
    {Thesis title: \textit{On the fitting scale dependence of the Parton Distributions}}
    {University of Milan} % Skills
    {2024}

    \cvhonor
    {TA for the course of Quantum Physics I} % Category
    {Introduction to Quantum Mechanics}
    {University of Milan} % Skills
    {2024}

    \cvhonor
    {TA for the course of Physics} % Category
    {Basics of Classical Mechanics and Thermodynamics}
    {University of Milan} % Skills
    {2024}

    \cvhonor
    {TA for the course of Quantum Physics II} % Category
    {Advanced course on Quantum Mechanics}
    {University of Milan} % Skills
    {2024}

    \cvhonor
    {TA for the course of Theoretical Physics I} % Category
    {Introduction to Quantum Field Theory}
    {University of Milan} % Skills
    {2023}

    \cvhonor
    {TA for the course of Physics} % Category
    {Basics of Classical Mechanics and Thermodynamics}
    {University of Milan} % Skills
    {2023}

    \cvhonor
    {TA for the course of Quantum Physics II} % Category
    {Advanced course on Quantum Mechanics}
    {University of Milan} % Skills
    {2023}

    \cvhonor
    {Exercise classes for the course of Quantum Physics II} % Category
    {Advanced course on Quantum Mechanics}
    {University of Milan} % Skills
    {2023}

    \cvhonor
    {TA for the course of Quantum Physics I} % Category
    {Introduction to Quantum Mechanics}
    {University of Milan} % Skills
    {2022}

    \cvhonor
    {\vspace{0.35cm}\href{https://www.uniroma1.it/en/pagina/student-collaboration-scholarships}{Student collaboration scholarship}} % Category
    {TA for the Laboratory courses of the first three years}
    {University of Rome ``La Sapienza''} % Skills
    {\vspace{0.35cm}2021}

    %---------------------------------------------------------
\end{cvhonors}
% %-------------------------------------------------------------------------------
%	SECTION TITLE
%-------------------------------------------------------------------------------
\cvsection{Others}


%-------------------------------------------------------------------------------
%	SUBSECTION TITLE
%-------------------------------------------------------------------------------
% \cvsubsection{International Awards}


%-------------------------------------------------------------------------------
%	CONTENT
%-------------------------------------------------------------------------------
\begin{cventries}

%---------------------------------------------------------
  \cventry
  {University of Rome ``La Sapienza'', SoRT} % Location
    {Student Collaboration Scholarship} % Event
    {Rome, Italy}
    {2021} % Date(s)
    {
      \begin{itemize} % Description(s) of tasks/responsibilities
        \item I won one of the 39 collaboration scholarships at the Physics department of the University of Rome 
        ``La Sapienza''.
        All informations can be gathered from the official page \url{https://www.uniroma1.it/en/pagina/student-collaboration-scholarships}.
      \end{itemize}
    }
%---------------------------------------------------------
\end{cventries}


%-------------------------------------------------------------------------------
\end{document}