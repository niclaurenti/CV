\cvsection{Experience}


%-------------------------------------------------------------------------------
%	CONTENT
%-------------------------------------------------------------------------------
\begin{cventries}

%---------------------------------------------------------

    \cventry
    {Embedded Software Developer at Next Ingegneria dei Sistemi S.p.A.}
    {Embedded Software Developer}
    {Rome, Italy}
    {Oct.\ 2024 - Current}
    {
        \begin{cvitems} % Description(s) of tasks/responsibilities
            \item Worked for \textbf{Next Ingegneria dei Sistemi} as a consultant software developer for \href{https://www.mbda-systems.com/country-it?rc=1}{\textbf{MBDA} Italy}, contributing to a complex defense-related embedded software project.
            \item Translated high-level system requirements into detailed low-level software specifications, developed software components written mainly in \texttt{C++}, \texttt{Java} and \texttt{Ada} and then designed and performed corresponding test procedures.
            \item Participated in system integration activities, interfacing and validating multiple software components across subsystems.
            \item Collaborated within a large, multi-company team, ensuring cross-organizational alignment and timely delivery of project milestones.
            \item[] \textbf{\textcolor{awesome-red}{Tec}hnologies}: \cpplogo{}~C++, \javalogo{}~Java, \adalogo{}~Ada, \qtlogo{}~Qt Creator, \netbeanslogo{}~Apache NetBeans, \gnatlogo{}~Gnat Studio, \SQLlogo{}~SQL, \bashlogo{}~Bash, \xmllogo{}~XML, \rtclogo{}~IBM RTC, \doorslogo{}~IMB Doors, \wiresharklogo{}~Wireshark, \linuxlogo{}~Linux, \windowslogo{}~Windows, \officelogo{}~Microsoft Office
        \end{cvitems}
    }

%---------------------------------------------------------
    
    \cventry
    {Researcher in Theoretical Particle Physics at the University of Milan and INFN}
    {Ph.D.\ Researcher}
    {Milan, Italy}
    {Oct.\ 2021 - Sept.\ 2024}
    {
        \begin{cvitems} % Description(s) of tasks/responsibilities
            \item Worked under the supervision of \href{https://inspirehep.net/authors/1009661?ui-citation-summary=true}{Prof.\ Stefano Forte} in the \href{https://nnpdf.mi.infn.it}{\textbf{NNPDF}} collaboration 
            as a developer of the \texttt{NNPDF} code \href{https://github.com/NNPDF}{\githublogo}.
            \item Developed techniques and computational programs that utilize artificial intelligence to 
            investigate the internal structure of the proton analysing experimental data collected at \href{https://home.cern}{\textbf{CERN}}.
            \item Developed programs for solving the so-called \href{https://eko.readthedocs.io/en/latest/theory/DGLAP.html}{DGLAP equations}, a linear system of integro-differential equations, with numerical techniques.
            \item Published research results in various papers and presented them in conferences.
            \item[] \textbf{\textcolor{awesome-red}{Tec}hnologies}: \pythonlogo{}~Python, \numpylogo{}~Numpy, \scipylogo{}~Scipy, \matplotliblogo{}~Matplotlib{}, \keraslogo{}~Keras, \tensorflowlogo{}~Tensorflow, \numbalogo{}~Numba, \fortranlogo{}~Fortran, \bashlogo{}~Bash, \gitlogo{}~Git, \githublogo{}~Github, \yamllogo{}~YAML, \jsonlogo{}~JSON, \slurmlogo{}~Slurm, \pbslogo{}~PBS, \mathematicalogo{}~Wolfram Mathematica, \linuxlogo{}~Linux, \faApple{}~MacOS, \vscodelogo{}~VS Code, \vimlogo{}~Vim, \latexlogo{}~Latex, \SQLlogo{}~SQL, \sqlitelogo{}~SQLite
        \end{cvitems}
    }

%---------------------------------------------------------

    \cventry
    {Researcher in Theoretical Particle Physics at the University of Rome ``La Sapienza''}
    {Undergraduate Researcher}
    {Rome, Italy}
    {Mar.\ 2021 - Oct.\ 2021}
    {
        \begin{cvitems} % Description(s) of tasks/responsibilities
            \item Worked under the supervision of \href{https://inspirehep.net/authors/1058479?ui-citation-summary=true}{Dr.\ Marco Bonvini} to develop theoretical methods and computational programs for producing high-precision theoretical predictions in particle physics.
            \item Focused on describing experimental data of electron-proton collisions, collected at the particle accelerators \href{https://en.wikipedia.org/wiki/HERA_(particle_accelerator)}{\textbf{HERA}} and \href{https://en.wikipedia.org/wiki/SLAC_National_Accelerator_Laboratory}{\textbf{SLAC}}.
            \item Wrote from zero the \texttt{C++} library \texttt{Adani} \href{https://github.com/niclaurenti/adani}{\githublogo}, with the \texttt{Python} bindings available in the \href{https://pypi.org/project/adani/}{PyPI} and in \href{https://anaconda.org/conda-forge/adani}{conda-forge}, resulting in a published paper and presentations at conferences.
            \item[] \textbf{\textcolor{awesome-red}{Tec}hnologies}: \cpplogo{}~C++, \gnulogo{}~GSL, \mathematicalogo{}~Wolfram Mathematica, \linuxlogo{}~Linux, \bashlogo{}~Bash, \cmakelogo{}~CMake, \emacslogo{}~Emacs, \latexlogo{}~Latex
        \end{cvitems}
    }

\end{cventries}