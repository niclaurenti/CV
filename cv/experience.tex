\cvsection{Experience}


%-------------------------------------------------------------------------------
%	CONTENT
%-------------------------------------------------------------------------------
\begin{cventries}

%---------------------------------------------------------
  \cventry
  {Researcher in Theoretical Particle Physics at the University of Milan and INFN}
  {Ph.D.\ Researcher}
  {Milan, Italy}
  {Oct.\ 2021 - current}
  {
      \begin{cvitems} % Description(s) of tasks/responsibilities
          \item Worked under the supervision of Prof.\ Stefano Forte in the \href{https://nnpdf.mi.infn.it}{\textbf{NNPDF}} collaboration 
          as a developer of the \texttt{NNPDF} code \href{https://github.com/NNPDF}{\faGithubSquare}.
          \item Developed techniques and computational programs applied to particle physics, that utilize artificial intelligence for 
          investigating the internal structure of the proton with high precision using experimental data collected at \href{https://home.cern}{\textbf{CERN}}.
          \item Published research results in various papers and presented them in conferences.
      \end{cvitems}
    }

    \cventry
{Researcher in Theoretical Particle Physics at the University of Rome ``La Sapienza''}
{Undergraduate Researcher}
{Rome, Italy}
{Mar.\ 2021 - Oct.\ 2021}
{
      \begin{cvitems} % Description(s) of tasks/responsibilities
        \item Worked under the supervision of Dr.\ Marco Bonvini with another Master student to develop theoretical methods and computational programs for producing high-precision theoretical predictions in particle physics.
        \item Focused on describing experimental data collected at the particle accelerator \href{https://en.wikipedia.org/wiki/HERA_(particle_accelerator)}{\textbf{HERA}}.
        \item Developed two programs, \texttt{Adani} \href{https://github.com/niclaurenti/adani}{\faGithubSquare} and \texttt{DIS\_TP} \href{https://github.com/andreab1997/DIS_TP}{\faGithubSquare}, resulting in a published paper and presentations at conferences.
      \end{cvitems}
    }


\end{cventries}