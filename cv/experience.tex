\cvsection{Experience}


%-------------------------------------------------------------------------------
%	CONTENT
%-------------------------------------------------------------------------------
\begin{cventries}

%---------------------------------------------------------
  \cventry
  {Researcher in Theoretical Particle Physics at the University of Milan and INFN}
  {Ph.D.\ Researcher}
  {Milan, Italy}
  {Oct.\ 2021 - current}
  {
      \begin{cvitems} % Description(s) of tasks/responsibilities
        \item During my Ph.D., I worked under the supervision of Prof.\ Stefano Forte in the \href{https://nnpdf.mi.infn.it}{NNPDF collaboration} as a developer of the \href{https://github.com/NNPDF}{NNPDF code}.
        My role involved developing techniques and computational programs applied to particle physics.
        The aim of the research project was to utilize artificial intelligence to investigate, with high precision,
        the internal structure of the proton analyzing experimental data collected at \href{https://home.cern}{CERN}.
        \item The results of the work have been published in three papers and have been presented in conferences.
        \item Furthermore, during this period, I worked as a Teaching Assistant and Lecturer for both Bachelor's and Master's courses
        and I co-supervised Bachelor and Master theses. 
      \end{cvitems}
    }

    \cventry
{Researcher in Theoretical Particle Physics at the University of Rome ``La Sapienza''}
{Undergraduate Researcher}
{Rome, Italy}
{Mar.\ 2021 - Oct.\ 2021}
{
      \begin{cvitems} % Description(s) of tasks/responsibilities
        \item During my Master Thesis I worked, under the supervision of Dr.\ Marco Bonvini and in collaboration with another Master student,
        to develop theoretical methods and computational programs to produce high-precision theoretical predictions in particle physics.
        These predictions aimed to describe experimental data collected at the particle accelerator \href{https://en.wikipedia.org/wiki/HERA_(particle_accelerator)}{HERA}.
        \item As a result of the work, two programs have been written, \href{https://github.com/niclaurenti/adani}{\texttt{Adani}} and \href{https://github.com/andreab1997/DIS_TP}{\texttt{DIS\_TP}},
        a paper has been published and the results have been presented in conferences.
      \end{cvitems}
    }


\end{cventries}